\documentclass{beamer}
\usepackage[utf8]{inputenc}
\usepackage{amsmath}
\usepackage{amsfonts}
\usepackage{amssymb}
\usepackage[frenchb]{babel}
\usepackage[T1]{fontenc}
\usepackage{lmodern}
\usepackage{graphicx}
\usepackage{enumerate}
\usepackage{stmaryrd}
\usepackage{wrapfig}
\usepackage{pgf}
\usepackage{tikz}

\newcommand\smt[1]{[\![#1]\!]}
\usepackage{bussproofs}

\usetikzlibrary{arrows,automata}

\usetheme{Warsaw}
\newtheorem{theo}[theorem]{Théorème}
\title{Geometry of Synthesis}
\author[Thomas Bourgeat]{Auteur : D. Ghica}
\date{Janvier 2013}

\begin{document}

\begin{frame}
  \titlepage
\end{frame}

\begin{frame}{Plan}
  \tableofcontents
\end{frame}

\section{Un langage pour les circuits}

\begin{frame}{Objectif}
Basic Syntactic Control of Inference :
\begin{itemize}
\item Langage dédié pour la synthèse de circuits
\item Style fonctionnel
\item Typé
\item Dont on sait donner la sémantique de programmes.
\end{itemize}
\end{frame}

\begin{frame}{Types}
\begin{itemize}
\item $\sigma := com | cell | exp$
\item $\theta := \sigma | \theta * \theta' | \theta \rightarrow theta'$
\end{itemize}
\end{frame}

\begin{frame}{Description}
\begin{itemize}
\item Une sorte de $\lambda$-calcul simplement typé.
\item constantes :
\begin{itemize}
\item $1 :: exp$
\item $0 :: exp$
\item $asg :: cell * exp \rightarrow com$
\item $der :: cell \rightarrow exp$
\item $seq :: com * com \rightarrow com$
\item $seq :: com * exp \rightarrow exp$
\item $op :: exp * exp \rightarrow exp$
\item $if :: exp * com * com \rightarrow com$
\item $while :: exp * com \rightarrow com$
\item $newvar :: (cell \rightarrow com) \rightarrow com$
\item $newvar :: (cell \rightarrow exp) \rightarrow exp$ 
\end{itemize}
\item \emph{Remarque :} Pas de récursion, ni de composition parallèle.
\end{itemize}
\end{frame}

\begin{frame}{Système de types.}
\begin{itemize}
\item \begin{prooftree}
\AxiomC{}
\RightLabel{(Id)}
\UnaryInfC{$x:\theta \vdash x:\theta$}
  \end{prooftree}

\item \begin{prooftree}
\AxiomC{$\Gamma \vdash M:\theta$}
\RightLabel{(Wk)}
\UnaryInfC{$\Gamma , x:\theta' \vdash M:\theta$}
\end{prooftree}

\item \begin{prooftree}
\AxiomC{$\Gamma , x:\theta' \vdash M:\theta $}
\RightLabel{(Intro$\rightarrow$)}
\UnaryInfC{$\Gamma \vdash \lambda x.M :\theta' \rightarrow \theta$}
\end{prooftree}

\item \begin{prooftree}
\AxiomC{$\Gamma \vdash M : \theta'$}
\AxiomC{$\Gamma \vdash N : \theta $}
\RightLabel{(Intro$*$)}
\BinaryInfC{$\Gamma \vdash (M,N):\theta'*\theta$}
\end{prooftree}

\end{itemize}
\end{frame}

\begin{frame}{Partage de ressources.}
\begin{itemize}
\item Application de fonctions :
\begin{prooftree}
\AxiomC{$\Gamma\vdash F : \theta' \rightarrow \theta$}
\AxiomC{$\Delta\vdash M : \theta'$}
\RightLabel{(Elim$\rightarrow$)}
\BinaryInfC{$\Gamma,\Delta \vdash F M : \theta $}
\end{prooftree}
\end{itemize}
\end{frame}

\section{Et sa sémantique?}

\subsection{Opérationnelle}

\begin{frame}
\begin{itemize}
\item Les réductions sont de la forme : $M,s\Downarrow T,s'$
\begin{itemize}
\item $s/s'$ sont des états : ils attribuent des valeurs booléennes aux variables libres 
\item $M$ un terme semi-clos (c'est à dire que les variables libres sont des
Cell.)
\item $T$ un terminal (0,1, skip ou une abstraction)
\end{itemize}
\pause
\item \emph{impure :} on a une sorte d'environnement, un peu comme des références.
\end{itemize}
\end{frame}

\begin{frame}
\begin{itemize}
\item Comme en $\lambda$-calcul, on a une sorte de $\beta$-réduction, puis des
règles pour les constantes.
\item 
\begin{prooftree}
\AxiomC{$M, s \Downarrow\lambda x.M',s$}
\UnaryInfC{$M M'',s\Downarrow M' [M''/x], s$}
\end{prooftree}
\item 

\begin{prooftree}
\AxiomC{$V,s\Downarrow v,s'$}
\UnaryInfC{$der(V)\Downarrow s'(v), s'$}
\end{prooftree}

\end{itemize}
\end{frame}


\subsection{Dénotationnelle.}

\begin{frame}{On veut des circuits.}
\begin{itemize}
\item Et si la sémantique d'un programme était un circuit.
\item On veut que:
\begin{itemize}
\pause
\item Le circuit soit correct
\pause
\item Il soit relativement conçis/optimisé.
\end{itemize}
\end{itemize}
\end{frame}

\section{Des circuits particuliers.}

\begin{frame}{Discussion informelle}
\begin{itemize}
\item Un circuit c'est :
\begin{itemize}
\item Des entrées/sorties : des canaux de communication.
\item Un comportement, qui lie l'état des sorties en fonction de celui des
entrées.
\end{itemize}
\pause
\item Un circuit poignée-de-main est 
\pause
poli :
\begin{itemize}
\pause
\item Peut recevoir des demandes
\item Peut emettre des demandes
\item Réponds aux demandes
\item Reçois les réponses de ses demandes. 
\pause
\end{itemize}
\item Donc des canaux \emph{Req}, et des canaux \emph{Ack}
\end{itemize}
\end{frame}

\begin{frame}{Catégorie HC}
\begin{itemize}
\item Soit HC la catégorie :
\begin{itemize}
\item \emph{Objets :} Sont des ports qui possèdent deux attributs qui ont chacun
deux formes : Input/Output et Req/Ack. 
\item \emph{Morphismes :} $f : A \rightarrow B$ sont les circuits dont :  
\begin{itemize}
\item Les entrées sont les sorties de A et les entrées de B
\item Les sorties sont les entrées de A et les sorties de B
\item Les Req sont les Req de A et de B
\item Les Ack sont les Ack de A et de B
\end{itemize}
\end{itemize}
\end{itemize}
\end{frame}

\begin{frame}{Composition dans HC}

\begin{figure}[h]
\center
\includegraphics[scale=0.6]{comp.jpg}
\end{figure}
\end{frame}

\begin{frame}{Identité}
\begin{figure}[h]
\center
\includegraphics[scale=0.6]{id.jpg}
\end{figure}
\end{frame}

\begin{frame}
\begin{itemize}
\item Le neutre pour la loi est l'ensemble vide de ports.
\item Le tenseur de deux objets est l'union disjointes des ports (en gardant les
spécificités). 
\end{itemize}
\end{frame}

\begin{frame}{Monoidale}
\begin{figure}[h]
\center
\includegraphics[scale=0.3]{mono.jpg}
\end{figure}

\end{frame}



\begin{frame}{Cloture dans HC}
\begin{itemize}
\item Ben les fonctions de A dans B, c'est des ports avec des labels, et un
comportement.
\pause
\item Donc dans $A\Rightarrow B$, ont mets les ports avec les labels.
\item $Eval::A1\otimes(A_2\Rightarrow B_1)\rightarrow B_2$ se fait
naturellement. On mets tous les ports en parallèle, $A_1$ lié avec $A_2$, $B_1$
avec $B_2$. Voir le dessin au tableau
\end{itemize}
\end{frame}


\subsection{Encore plus particuliers.}
\begin{frame}
\begin{itemize}
\item On a une catégorie monoidale close
\pause
\item On veut un produit en plus.
\pause
\begin{theo}
Si on a une catégorie monoidale close, et que l'on est capable d'en trouver une
sous-catégorie qui vérifie que l'unité pour le produit monoidale est un objet
terminal, et que chaque objet a un morphisme diagonal, alors cette
sous-catégorie est cartésienne.
\end{theo}
\pause
\item Humhum, une catégorie cartésienne close, du $\lambda$-calcul, on va
pouvoir interpréter!
\end{itemize}
\end{frame}

\begin{frame}{Circuit SHC}
\begin{itemize}
\item Un \emph{circuit poignée-de-main simple} est \pause très poli :
\pause
\begin{itemize}
\item Entrées et sorties alternent
\item Les Req/Ack sont bien parenthésés. (i.e : on fait pas \texttt{Req1;Req2;Ack1;Ack
2})
\item On a pas de Req successifs sur le même port, sans réponse :
\texttt{Req1;Req1;Ack1;Ack1}
\pause
\item (La première requête est faite sur un port initial) 
\end{itemize}
\end{itemize}
\end{frame}

\begin{frame}
\begin{itemize}
\item Et si on considère de tels circuits, on a des circuits diagonaux comme
suit : 
\pause
\item $A \rightarrow A_1*A_2$ donc on a les ports de A dupliqués 3 fois, nommés

$A_i$. Puis se souvenir de i.
\item Si il y a une sortie sur $A$, produire une sortie aussi sur le port
correspondant de $A_i$, avec le i mémorisé.
\end{itemize}
\end{frame}

\subsection{Modules élémentaires.}

\begin{frame}{Alors, comment on interprète?}
\begin{itemize}
\item On a trois objets/types de bases :
\begin{itemize}
\item $\smt{com}=\{R\rightarrow(ir),D\rightarrow(oa)\}$
\item $\smt{exp}=\{Q\rightarrow(ir),T\rightarrow(oa),F\rightarrow(oa)\}$
\item $\smt{cell}=\{WT\rightarrow(ir),WF\rightarrow(ir),Q\rightarrow(ir),D\rightarrow(oa),T\rightarrow(oa),F\rightarrow(oa)\}$
\end{itemize}
\end{itemize}
\end{frame}

\begin{frame}

\begin{figure}[h]
\center
\includegraphics[scale=0.6]{const.jpg}
\end{figure}

\end{frame}

\begin{frame}

\begin{figure}[h]
\center
\includegraphics[scale=0.5]{if.jpg}
\end{figure}


\end{frame}
\section{Conclusion}

\begin{frame}
Conclusion :
\begin{itemize}
\item Premier article d'une série d'au moins 4
\pause
\item Récursivité
\pause
\item Concurrence 
\end{itemize}
\end{frame}

\begin{frame}
\begin{itemize}
\item Handshake Circuits an Intermediary Between Communicating Processes and
VLSI - K. Van Berkel
\item D.Ghica - Geometry of Synthesis
\end{itemize}
\end{frame}

\end{document}

