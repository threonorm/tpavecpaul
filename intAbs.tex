\documentclass{beamer}
\usepackage[utf8]{inputenc}
\usepackage{amsmath}
\usepackage{amsfonts}
\usepackage{amssymb}
\usepackage[frenchb]{babel}
\usepackage{lmodern}
\usepackage{graphicx}
\usepackage{enumerate}
\usepackage{stmaryrd}
\usepackage{wrapfig}
\usepackage{pgf}
\usepackage{tikz}


\newcommand\smt[1]{[\![#1]\!]}
\usetheme{Warsaw}
\newtheorem{theo}[theorem]{Théorème}
\title{Tracing compilation by abstract
interpretation.}
\author[Thomas Bourgeat]{Auteurs : Stefano Dissegna, Francesco Logozzo, Francesco Ranzato}
\date{Janvier 2013}

\begin{document}

\begin{frame}
  \titlepage
\end{frame}

\begin{frame}{Plan}
  \tableofcontents
\end{frame}

\section{Compilation à la volée.}

%On veut optimiser localement
% -est-ce que c'est correct?
% -Ou est-ce qu'on fait les optims? 

\subsection{Introduction}

\begin{frame}{Idées reçues}
\begin{itemize}
\item Python, Javascript sont lents.
\begin{itemize}
\item le code Python est compilé en bytecode, et exécuté par une machine
virtuelle.
\end{itemize}
\item C est rapide
\pause
\item C'est faux : un langage n'est pas lent. Son compilateur produit du code
lent.
\pause
\item Le C a été beaucoup étudié, et on sait faire beaucoup d'optimisations
qui font que le code produit est rapide. 
\item Le C a des caractéristiques intrinsèques qui facilitent ce genre
d'optimisation.
\end{itemize}
\end{frame}

\begin{frame}{Langage dynamique}
\begin{itemize}
\item Le bytecode est à priori plus lent que du code natif.
\item Le fait que Python soit dynamique le rend difficile à optimiser par les
analyses statiques.
\pause
\item Idée : Certaines \emph{zones chaudes} du programme pourraient être
dynamiquement compilées vers du code natif et exécutées.
\end{itemize}
\end{frame}

\begin{frame}{JIT compilers}
\begin{itemize}
\item Compilateurs à la volée : 
\begin{itemize}
\item \emph{Basiquement:} Lorsque l'"interpréteur" s'aperçoit qu'une fonction
est exécutée super souvent (dynamiquement). Il compile cette fonction
\item Pour les prochains appels, l'interpréteur utilisera directement le code
natif $\rightarrow$ il n'interpretera plus cette partie.
\end{itemize}
\end{itemize}
\end{frame}

\begin{frame}{Tracing JIT compilation}
\begin{itemize}
\item Pourquoi pas aller plus loin : 
\pause
\begin{itemize}
\item Compiler en natif et optimiser pas seulement les fonctions.
\item La machine passe beaucoup de temps dans des boucles : cherchons-les pour
les optimisers.
\end{itemize}  
\end{itemize}
\end{frame}

\begin{frame}{Problèmes}
\begin{itemize}
\item Repérer quelles sont les \emph{zones chaudes} (ou \emph{hot path})
\item Les optimisers
\item Prouver qu'on a pas fait n'importe quoi :
\begin{itemize}
\item Utiliser des optimisations prouvées correctes sous certaines hypothèses,
qu'on garantie être vérifiées quand les le code optimisé est exécuté
\item Donc garantir que le programme transformé \emph{se comporte} comme celui de
départ.
\end{itemize}
\end{itemize}
\end{frame}

\begin{frame}{L'interprétation abstraite dans l'histoire}
\begin{itemize}
\item Définir honnêtement: \emph{zone chaude}, \emph{se comporte}.
\item Donner une méthode pour trouver les zones chaudes.
\item Permettre des définitions modulaires. 
\item \emph{$\dots$} (trois petits points pipologiques)
\end{itemize}
\end{frame}


\subsection{Un langage qui décrit le flot.}


\begin{frame}
\begin{figure}[h]
\center
\includegraphics[scale=0.6]{syntax.jpg}
\end{figure}
\end{frame}

\begin{frame}
\begin{itemize}
\item Sémantique de trace :
\begin{itemize}
\item Environnement : $\rho: Var \rightarrow Val$
\item Commande : $ L : A \rightarrow L$
\end{itemize} 
\end{itemize}
\end{frame}

\begin{frame}
\begin{figure}[h]
\center
\includegraphics[scale=0.6]{ex1.jpg}
\end{figure}
\end{frame}

\begin{frame}
\begin{figure}[h]
\center
\includegraphics[scale=0.6]{ex2.jpg}
\end{figure}
\end{frame}

\section{Et dans le cadre de l'interprétation abstraite.}

\subsection{Programmes et comportements.}
\begin{frame}
\begin{itemize}
\item Définissons pour ce langage, les notions de comportement et de zones
chaudes.
\pause
\item $P$ \emph{se comporte} comme $P'$ ("au sens de $\alpha$", qui est une
abstraction de la sémantique de
trace):
\begin{itemize}
\item $\alpha(T\smt{P})=\alpha(T\smt{P'})$
\end{itemize}
\pause
\item \emph{zones chaudes} d'ordre n du programme P d'une trace $\sigma$: 
\begin{itemize}
\item Une trace abstraite de longueur n $(a_1,C_1),\dots,(a_n,C_n)$.
\item Il existe $(\rho_i,C_i),\dots(\rho_n,C_n) \in loop(\sigma)$ tels que 
$alpha((\rho_i,C_i),\dots(\rho_n,C_n))= (a_1,C_1),\dots,(a_n,C_n)$ et que la
suite des états réels apparait au moins n fois dans la trace $\sigma$
\item où $loop(\sigma)$ est l'ensemble des bouts de trace linéaires. (plus
petite détails)
\end{itemize}
\pause
\item \emph{Remarque:} On sait faire les calculs.
\end{itemize}
\end{frame}

\begin{frame}
\begin{itemize}
\item Pour l'équivalence comportementale, on prend
usuellement les changements d'environnements successifs. 
\item On pourrait aussi prendre seulement ce qu'il se passe sur la sortie
standard par exemple.
\end{itemize}
\end{frame}


\begin{frame}{Exemple}
\begin{itemize}
\item En prenant pour abstraction d'environnement un unique point :
$Store^\sharp=\{\top\} $
\item
\begin{figure}
\center
\includegraphics[scale=0.6]{hp.jpg}
\end{figure}
\end{itemize}
\end{frame}

\subsection{Décider quand exécuter le code optimisé.}
\begin{frame}{Une transformation de programme.}
\begin{itemize}
\item Le code optimisé dans la zone chaude doit-être utilisé uniquement si on
est certain que l'on a le droit.
\item On rajoute donc des gardes entre chaque actions, qui vérifient qu'on a
le droit de faire cette exécution optimisée. Sinon on sort du code optimisé pour
retourner vers la version safe.
\pause
\item A ce moment là, rappeller que tout ceci est fait dynamiquement. 
\end{itemize}
\end{frame}

\begin{frame}
\begin{figure}
\center
\includegraphics[scale=0.6]{optim.jpg}
\end{figure}
\end{frame}

\subsection{Spécialisation de types : controler les optimisations.}

\begin{frame}
\begin{itemize}
\item Certaines optimisations peuvent ne marcher que sous certaines hypothèses
locales. Exemple : avoir des entiers dans certaines variables.
\item On veut donc  extraire de l'information
et ainsi essayer de faire des optimisations plus ciblées :
\begin{itemize}
\item Il faut que ce soit dynamique : fait à l'exécution par ce le monitor de
trace.
\pause
\item Une idée naturelle est de regarder du côté du typage.
\end{itemize}
\end{itemize}
\end{frame}

\begin{frame}
\begin{itemize}
\item On choisi l'"abstraction de type" pour chercher les zones chaudes :
\begin{itemize}
\item Un treillis des types.
\item $\alpha_{type} : P(Value) \rightarrow Types$
\item Se relève sur les environnements, et donc c'est gagné.
\end{itemize} 
\item Cette fois-ci les gardes sont plus utiles que l'exemple précèdent. Passer
une garde, c'est garantir que certaines variables contiennent des valeurs d'un
certain type. 
\end{itemize}
\end{frame}


\section{C'est mieux qu'avant.}

\begin{frame}{Travaux précèdents}
\begin{itemize}
\item Dans les travaux précèdents :
\begin{itemize}
\item Un langage de style impératif.
\item Une sémantique à petit pas.
\item Bisimulation (orientée jeu) pour parler d'équivalences comportementales.
\end{itemize}
\pause
\item Ce qui existait peut se plonger dans le framework proposé par l'article.
\begin{itemize}
\item Une fonction de compilation.
\item une sémantique de trace sur les programmes de Guo et Palsberg
\item Une compilation correcte : $\alpha_{st}(T_{GP}\smt{S} ) =
\alpha_{st}(T\smt{C(S)})$.
\item Théorème d'équivalence entre  bisimulation et abstraction de traces identique.  
\end{itemize}
\pause
\item Mais en plus on a un exemple du fait que le cadre via IA est strictement
mieux!
\end{itemize}
\end{frame}


\begin{frame}{Conclusion}
\begin{itemize}
\item Pour conclure :
\begin{itemize}
\item On a un cadre pour détecter dynamiquement des zones de code qui se
repètent en vérifiant certaines propriétés.
\item On a une manière d'être sur qu'on fait pas des optimisations quand il faut
pas.
\item On a gagné une manière d'exprimer des équivalences comportementales.
\end{itemize}
\end{itemize}
\end{frame}

\begin{frame}
\begin{itemize}
\item P. Cousot and R. Cousot. \emph{Systematic design of program transformation
frameworks by abstract interpretation.} In Proceedings of the 29th
ACM SIGACT-SIGPLAN Symposium on Principles of Programming
Languages (POPL 2002), pages 178–190, New York, NY, USA, 2002.
ACM.

\item 
Stefano Dissegna, Francesco
Logozzo, and Francesco Ranzato. \emph{Tracing compilation by abstract
interpretation.} In Principles of Programming Languages (POPL
2014).


\end{itemize}
\end{frame}
\end{document}

