\documentclass[10pt,a4paper]{article}
\usepackage[francais]{babel}  %Doc fr
\usepackage{fullpage}
\usepackage{euler}
\usepackage{fontspec}
\usepackage{amsmath}
\usepackage{framed}
\usepackage{amsfonts}
\usepackage{amssymb}
\usepackage{graphicx}
\usepackage{hyperref}

\setmainfont[Numbers=OldStyle]{Linux Libertine O}

\begin{document}
\title{Caml-TD1 : Quelques jeux}
\author{Thomas Bourgeat\footnote{Vous pouvez bien entendu me contacter par mail thomas.bourgeat@ens.fr, même en cas de début de question : N'HÉSITEZ PAS.}\and Paul Melotti}
\maketitle{}

\section{Nombres de Fibonacci et mémoïzation}
Soit $(u_n)$ définie par $u_0=u_1=1$ et $u_{n+2}=u_{n+1}+u_n$
\paragraph{Q.1} Programmer une fonction récrusive \texttt{fibo : int-> int} qui calcule $u_n$. Quelle est la complexité de votre fonction? A quoi pouvez-vous penser pour l'améliorer.

Dans cette première partie nous allons proposer une manière un peu différente de faire de la récursivité : \emph{la mémoïzation}.

Une table de hachage est un tableau de liste : \texttt{table : (a'*b')
list array} ainsi qu'une fonction de hachage \texttt{h: 'a -> int}. Le
but est de stocker des éléments de type b, indexés par leurs clefs de
type a. Pour stocker $(k,e)$ dans une table,on calcule le hachage h(k),
qui est l'indice du tableau auquel on va mettre en tête de liste $(k,e)$. De même on peut effectuer une recherche ou une suppression en temps raisonnable.
Toutes les manipulations portent sur les clefs et le hachage des clefs.
L'idée de la mémoization consiste à stocker les $(n,f(n))$ d'une fonction
récursive au fur et à mesure qu'on les calcule. Ainsi si l'on a besoin plus tard
d'un calcul déjà effectué il est inutile de le refaire  : on va le
chercher dans la table.


\paragraph{Q.1}Quelles sont les complexités de recherche, d'insertion, de
suppression? Quelle propriété faut-il sur h pour que notre table de
hachage soit intéressante.

\paragraph{Q.2} Implémenter une table de hachage \texttt{table} de taille
17. Avec la fonction de hachage qui consiste simplement à réduire
l'entrée qui est un entier modulo 17. On programmera une fonction
d'ajout, une fonction de recherche à partir de la clef (qui renvoit
l'élément cherché s'il est dans la table, lève une exception sinon),
et une fonction de suppression.


La mémoïzation consiste à faire de la récursivité améliorée : lorsqu'on appelle la fonction sur un élément plus petit on vérifie d'abord si on ne l'aurait pas déjà calculé, auquel cas on le renvoit immédiatement. Sinon on le calcul, on le rajoute dans la table puis on le renvoit. Les éléments déjà calculés sont stockés dans une table de hachage.

\paragraph{Q.3} Implémenter une fonction qui calcule Fibonacci en utilisant la mémoïzation.
 

\section{Problème de Joseph}
$n$ personnes se réunissent en cercle dans une salle. Ils sont menacés par des ennemis à l'extérieur et décident donc de se suicider, pour sauver leur honneur, chacun leur tour. Pour ça ils choisissent un premier, et en partant de là comptent m personnes, et cette personne doit se suicider. Puis on recommence avec une personne de moins, à partir de la personne qui suivait la dernière morte. Le but est d'être le dernier survivant : certainement pour retourner sa veste et peut-être s'en sortir. Pour plus d'histoire on ira voir la page wikipédia de Flavius Joseph.

\paragraph{Q.1} Proposer une manière de représenter une liste circulaire. 

\paragraph{Q.2} Écrire une fonction qui transforme une liste en liste circulaire.

\paragraph{Q.3} Écrire une fonction de recherche d'élément dans une liste circulaire.

\paragraph{Q.4} Écrire une fonction de suppression d'élément dans une liste circulaire.

\paragraph{Q.5} En déduire une fonction \texttt{joseph: int -> int -> int} qui prend deux arguments n et m et détermine quelle est la position gagnante
lorsqu'il y a n personnes dans le cercles et qu'on en tue une sur m.


\subsection{Version cours de récréation de l'école primaire de Paul Melotti.}
La problèmatique, non moins sérieuse est légèrement différente : on souhaite désigner le loup\footnote{Bien entendu tout le monde sait jouer au loup} parmi un groupe d'enfants.
Cette fois-ci on ne reprends pas à compter à partir du survivant qui suit le dernier mort, mais toujours en repartant du premier survivant qui suit le compteur. Si le compteur est éliminé, son premier successeur devient compteur.
Et accessoirement on ne tue personne : c'est la version kikoolol. Le dernier restant est le loup.

On numérote les $n$ enfants de $0$ à $n-1$, le compteur initial portant le numéro $0$. On note $l(n,m)$ le loup élu pour $n$ enfants avec une comptine de longueur $m$.

\paragraph{Q.1} Montrer que $l(n,m) = \left\{
	\begin{array}{ll}
		0  & \mbox{si } n = 1 \\
		l(n-1,m) & \mbox{si } l(n-1,m) < m \% n \\
		l(n-1,m) + 1 & \mbox{sinon}.
	\end{array}
\right. $
\paragraph{Q.2} Écrire une fonction qui résout ce problème. On pourra soit refaire comme dans la partie précédente, soit utiliser la question 1.
\paragraph{Q.3} Montrer que $l(n-1,n) = 0$ ssi $n$ est premier.
\paragraph{Q.4} Ce jeu vous semble-t-il juste? Y-a-t-il des positions préférentielles à $n$ fixé? (On pourrait imaginer qu'on tire $m$ aléatoire au début du jeu. Faire des expérience avec notre fonction.)
\subsection{Pour aller plus loin}
\paragraph{Q.1}Proposer une implémentation fonctionnelle des listes circulaires.
\paragraph{Q.2}Combien y-a-t-il de positions gagnantes au Tic Tac Toe
3x3x3. 
\section{Références}
Le problème de Joseph est très bien étudié dans le livre \emph{Concrete Mathematics}. La récurrence est résolue dans le cas $m=2$ de plusieurs manières différentes.

\end{document}
