\documentclass[10pt,a4paper]{article}
\usepackage[francais]{babel}  %Doc fr
\usepackage{fullpage}
\usepackage{euler}
\usepackage{fontspec}
\usepackage{amsmath}
\usepackage{framed}
\usepackage{amsfonts}
\usepackage{amssymb}
\usepackage{graphicx}
\usepackage{hyperref}

%\setmainfont[Numbers=OldStyle]{Linux Libertine O}

\begin{document}
\title{Introduction aux graphes : des relations binaire aux systèmes de votes.}
\author{Thomas Bourgeat\footnote{Vous pouvez bien entendu me contacter par mail thomas.bourgeat@ens.fr, même en cas de début de question : N'HÉSITEZ PAS.}\and Paul Melotti}
\maketitle{}

\section{Graphe de relation}

Le but du sujet est de vous faire griffoner un nombre maximum de petits
dessins. Vous êtes donc conviés à sortir une feuille et à réfléchir avec
ce nouvel objet qu'est le graphe. 
Dans ce sujet on va s'intéresser à visualiser graphiquement des
propriétés de relations. Pour représenter une relation binaire $\mathcal{R}$ sur un ensemble
$X$, on va construire le graphe suivant : 
\begin{itemize}
\item Les sommets sont les éléments de l'ensemble.
\item On construit une arête (orientée) $(s,t)$ lorsque $(s,t)\in
\mathcal{R}$
\end{itemize}
\paragraph{Question. 1\\}
On implémente une relation binaire sur un ensemble fini X, par son graphe
comme définit plus haut. Et on représente en Caml ce graphe par sa
matrice d'adjacence. Construisez le graph correspondant à la relation de
divisibilité entière sur l'ensemble $\{1,2,\dots,9\}$

\paragraph{Question. 2\\}
Décrire quelles propriétés particulières vérifie le graphe lorsque la
relation est :
\begin{itemize}
\item Réflexive
\item Symétrique
\item Transitive
\item Antisymétrique
\end{itemize}  
Écrire des fonctions qui étant donné un graphe de relation, vérifient si
la relation associée est Réflexive, Symétrique\dots

\paragraph{Question. 3\\}
Si on part d'une relation d'équivalence. Comment se manifeste une classe
d'équivalence dans le graphe. Écrire une fonction qui étant donné une
relation d'équivalence, renvoit l'ensemble des classes d'équivalence
(comme une liste de listes).

\paragraph{Question. 4\\}
Étant donné une relation $\mathcal{R}$ on appelle la cloture refléxive (resp. transitive,
symétrique \dots), la plus petite relation réflexive (resp. transitive,
symétrique \dots) qui contient $\mathcal{R}$. Écrire des fonctions qui
étant donné une relation, renvoient la cloture transitive, symétrique\dots

\paragraph{Question. 5\\}
A quoi ressemble un ordre total\,? On appelle linéarisation d'un ordre
partiel, un ordre total qui contient cet ordre partiel. Écrire une
fonction qui linéarise un ordre partiel donné en entrée.
 
\section{Paradoxe de Condorcet}

\paragraph{Question. 1\\}

\paragraph{Question. 2\\}
La relation obtenue est-elle un ordre partiel\,? Est-elle transitive\,?
Y-a-t-il un élément maximum\,?

\paragraph{Question. 3\\}
Proposer une manière de résoudre le paradoxe de Condorcet.

\section{Morphisme de graphes}
On appelle un morphisme de graphe, une application d'un graphe dans un
autre qui préserve l'adjacence.

\paragraph{Question. 1\\} Ecrire une fonction qui prends deux graphes,
une fonction d'un graphe dans l'autre, et détermine si cette fonction est
un morphisme de graphe. 

\paragraph{Question. 2\\}
Qu'est-ce qu'un morphisme de graphes pour le monde des relations d'ordres. 

\paragraph{Facultatif culturel\,:\\} Écrire une fonction qui détermine si deux graphes
sont isomorphes. Quelle est la complexité de votre algorithme\,?

\section{Coloriage de graphes}

Un k-coloriage d'un graphe G est un coloriage des sommets du graphe G à
l'aide de k couleurs, tel que deux sommets adjacents ne sont jamais de la
même couleur.

\paragraph{Question. 1\\}
Montrer que tout arbre est 2-coloriable.

\paragraph{Question. 2\\}
On va encoder un coloriage par un tableau d'entier, de la taille du
nombre de sommets du graphe. Montrer que tout graphe de degrés borné par
$k$, est $k+1$-coloriable. Le degrés d'un sommet est le nombre de voisin
de ce sommet, le degrés d'un graphe et le maximum des degrés de ses
sommets. Écrire une fonction qui renvoit un tel coloriage. 

\paragraph{Question. 3\\}
On définit le nombre chromatique du graphe comme le plus petit $k$ tel que
G soit $k$-coloriable. Si c'est le graphe d'une relation d'equivalence, à quoi
correspond ce k\,?

\section{Une dernière notion : Circuits Hamiltoniens et Eulériens}
Ici on va considérer des graphes non-orientés, connexe.
On appelle un circuit Hamiltonien (resp. Eulérien) un chemin dans un graphe qui passe une
et une seule fois par chaque sommet (resp. arête). Contrairement à votre
intuition légitime de deux problèmes de complexité semblable. On pense
que l'un est beaucoup plus difficile que l'autre. 

\paragraph{Question. 1\\} Écrire une fonction \texttt{isHamiltonian} qui
décide si un graphe est hamiltonien, et si oui renvoit un tel chemin.
Quelle est la complexité de votre algorithme\,? \emph{Culture\,: C'est le
bien connu problème du voyageur de commerce.}

\paragraph{Question. 2\\}
On appelle code de Gray une énumération des entiers de $[0,2^n-1]$ par
des mots binaires de n bits, telle que deux entiers successifs ont des
représentants qui ne diffèrent que d'un bit. En utilisant la question
précèdente sur un graphe bien choisit,  montrez l'existence d'un tel code
de Gray pour $n=4$. Ceci était une $\epsilon-$introduction à la théorie
des codes correcteurs d'erreurs. C'est \emph{*intéressant*}. Certains
code de Gray sont simples à calculer\,: on pourra prendre 2mn pour se
documenter.

\paragraph{Question. 3\\} Prouvez qu'un graphe connexe non orienté (d'un seul tenant) possède un circuit
Eulérien si et seulement si chaque sommet est incident à un nombre pair
d'arêtes. Écrire une fonction Caml \texttt{isEulerian} qui décide si un graphe possède un
cycle eulérien. 

\end{document}
