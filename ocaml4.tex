\documentclass[10pt,a4paper]{article}
\usepackage[francais]{babel}  %Doc fr
\usepackage{fullpage}
\usepackage{euler}
\usepackage{fontspec}
\usepackage{amsmath}
\usepackage{framed}
\usepackage{amsfonts}
\usepackage{amssymb}
\usepackage{graphicx}
\usepackage{hyperref}
\usepackage{enumerate}

%\setmainfont[Numbers=OldStyle]{Linux Libertine O}

\begin{document}
\title{Introduction aux graphes~: des relations binaires aux systèmes de votes}
\author{Thomas Bourgeat\footnote{Vous pouvez bien entendu me contacter par mail thomas.bourgeat@ens.fr, même en cas de début de question : N'HÉSITEZ PAS.}\and Paul Melotti}
\maketitle{}

Les questions marquées d'une étoile sont plus difficiles et pourront être sautées.

\section{Graphe de relation}

Le but du sujet est de vous faire griffoner un nombre maximum de petits
dessins. Vous êtes donc conviés à sortir une feuille et à réfléchir avec
ce nouvel objet qu'est le graphe. 
Dans ce sujet on va s'intéresser à visualiser graphiquement des
propriétés de relations. Pour représenter une relation binaire $\mathcal{R}$ sur un ensemble
$X$, on va construire le graphe suivant~: 
\begin{itemize}
\item Les sommets sont les éléments de l'ensemble.
\item On construit une arête (orientée) $(s,t)$ lorsque $s \mathcal{R} t$.
\end{itemize}
On note $\mathcal{G}_{\mathcal{R}}$ ce graphe.
\paragraph{Question. 1\\}
On représente une relation binaire sur un ensemble fini $X$ par son graphe.
On représente en Caml ce graphe par sa
\textit{matrice d'adjacence}~: c'est une matrice de booléens de taille 
$|X| \times |X|$, telle que l'indice $s,t$ vaut \texttt{true} ssi 
$s\mathcal{R} t$. On notera \texttt{graphe} son type.

Dessinez le graphe et écrivez la matrice d'adjacence pour les relations 
suivantes~:
\begin{itemize}
\item divisibilité entière sur l'ensemble $X=\{1,2,\dots,6\}$~;
\item inclusion ensembliste sur les parties de $\{\mathrm{riri};\mathrm{fifi};
\mathrm{loulou}\}$.
\end{itemize}

\paragraph{Question. 2\\}
Écrire des fonctions~:
\begin{enumerate}[a)]
\item \texttt{complet: int -> graphe} qui étant donné le cardinal $n$ de $X$
renvoie le graphe de la \textit{relation complète} sur $X$, notée $\mathcal{C}$,
définie par~: $\forall x, y, \ x \mathcal{C} y$.
\item \texttt{diag: int -> graphe} qui étant donné le cardinal $n$ de $X$
renvoie le graphe de la \textit{relation diagonale} sur $X$, notée $\Delta$, 
définie par~: $x \Delta y \ \mathrm{ssi} \ x=y$.
\item \texttt{inter: graphe -> graphe -> graphe} qui étant donné
deux graphes $\mathcal{G}$ et $\mathcal{G'}$ ayant les mêmes sommets renvoie leur 
\textit{intersection}, c'est-à-dire le graphe dont les arêtes sont celles qui appartiennent
à la fois à $\mathcal{G}$ et à $\mathcal{G'}$.
\item \texttt{union: graphe -> graphe -> graphe} qui étant donné
deux graphes $\mathcal{G}$ et $\mathcal{G'}$ ayant les mêmes sommets renvoie leur 
\textit{union}, c'est-à-dire le graphe dont les arêtes sont celles qui appartiennent à
$\mathcal{G}$ ou à $\mathcal{G'}$.
\item \texttt{inclus: graphe -> graphe -> bool} qui étant donné
deux graphes $\mathcal{G}$ et $\mathcal{G'}$ ayant les mêmes sommets dit si 
$\mathcal{G} \subset \mathcal{G'}$, c'est-à-dire si toute arête de $\mathcal{G}$
est arête de $\mathcal{G'}$.
\item \texttt{recip: graphe -> graphe} qui étant donné un graphe 
$\mathcal{G}$ renvoie sa \textit{réciproque}, c'est-à-dire le graphe qui contient l'arête 
$(s,t)$ ssi $\mathcal{G}$ contient l'arête $(t,s)$.
\item \texttt{compose: graphe -> graphe -> graphe} qui étant donné 
deux graphes $\mathcal{G}$ et $\mathcal{G'}$ ayant les mêmes sommets renvoie leur 
\textit{composée}, c'est-à-dire le graphe qui contient l'arête $(s,t)$ ssi il existe 
$u$ tel que $(s,u)\in \mathcal{G}$ et $(u,t) \in \mathcal{G'}$.
\end{enumerate}

\paragraph{Question. 3\\}
Décrire en termes d'union, d'intersection et d'inclusion de graphes les 
propriétés suivantes sur la relation~:
\begin{itemize}
\item Réflexive ($\forall x, \ x\mathcal{R}x$)
\item Symétrique ($\forall x,y, \ x\mathcal{R}y \implies y\mathcal{R}x$)
\item Transitive ($\forall x,y,z, \ x\mathcal{R}y \ \mathrm{et} \ y\mathcal{R}z \implies x\mathcal{R}z$)
\item Antisymétrique ($\forall x,y, \ x\mathcal{R}y \ \mathrm{et} \ y\mathcal{R}x \implies x=y$)
\item Totale ($\forall x,y, \ x\mathcal{R}y \ \mathrm{ou} \ y\mathcal{R}x$)
\end{itemize}  
Écrire des fonctions qui étant donné le graphe de la relation, testent ces 
différentes propriétés.

\paragraph{Question. 4\\}
\begin{enumerate}[a)]
\item Étant donné une relation $\mathcal{R}$ on appelle la clôture refléxive
la plus petite relation réflexive qui contient $\mathcal{R}$. Écrire une 
fonction qui étant donné le graphe d'une relation renvoie le graphe de sa 
clôture réflexive.
\item Même question pour ``symétrique''.
\item (*) Même question pour ``transitive''.
\end{enumerate}

\paragraph{Question. 5\\}
On suppose dans cette question que $\mathcal{R}$ est une relation d'équivalence.
Comment se manifeste une classe
d'équivalence dans $\mathcal{G}_{\mathcal{R}}$~? Écrire une fonction qui étant 
donné le graphe d'une relation d'équivalence, renvoie l'ensemble des classes 
d'équivalence (au choix comme une liste de listes ou comme un tableau de listes).

\paragraph{Question. 6\\}
(*) À quoi ressemble un ordre total~? On appelle linéarisation d'un ordre
partiel, un ordre total qui contient cet ordre partiel, au sens de l'inclusion
des graphes. Écrire une
fonction qui linéarise un ordre partiel.
 
\section{Paradoxe de Condorcet}
On considère le système de vote suivant~: 3 candidats,
60 votants, et on demande à chaque votant de donner une liste ordonnée de ses 
préférences.
\paragraph{Question. 1\\}
Considérons la situation         
\begin{itemize}
\item 23 votants préfèrent\,: A > B > C
\item 17 votants préfèrent\,: B > C > A
\item 2 votants préfèrent\,: B > A > C
\item 10 votants préfèrent\,: C > A > B
\item 8 votants préfèrent\,: C > B > A
\end{itemize}
Construire le graphe de la relation suivante~: on considère que X est moins 
bien que Y ($X<Y$) si il y a une majorité de gens qui pensent que X 
est moins bien que Y.
\paragraph{Question. 2\\}
La relation obtenue est-t-elle~:
\begin{itemize}
\item avec un élément maximum~?
\item un ordre partiel~?
\item transitive~?
\end{itemize}
\paragraph{Question. 3\\}
Proposer une manière de résoudre le paradoxe de Condorcet. Qui proposez-vous 
d'élire~?

\section{Une solution~: le scrutin par rangement des paires}
On souhaite appliquer l'idée de vote de Condorcet dans le cas général.
Comme précédemment, on construit le graphe de la relation ``une majorité de 
personnes pense que $A<B$''. On suppose que 
le nombre d'électeurs est impair, de manière à toujours avoir l'arête 
$(A,B)$ ou bien l'arête $(B,A)$.

Le problème, on l'a vu, est que ce graphe peut contenir des cycles (dans un 
graphe, un cycle est une suite $c_0,... c_n$ de sommets tels que $(c_0,c_1),...,
(c_{n-1},c_n), (c_n,c_0)$ sont des arêtes du graphe). Pour éviter ce problème, 
on va pondérer l'arête $(A,B)$ par le nombre de personnes qui pensent que $A<B$.
Au lieu d'ajouter toutes les arêtes d'un coup, on les 
ajoute par ordre de poids décroissant, en n'ajoutant pas les 
arêtes qui créent des cycles. À la fin, on obtient un graphe sans cycle 
(\textit{acyclique}), et on élit l'élément maximum de ce graphe 
(on va montrer que cet élément existe et est unique).

\paragraph{Question. 1\\}
Appliquer cette méthode sur l'exemple précédent.

\paragraph{Question. 2\\}
Montrer que dans un graphe fini acyclique, il existe un élément 
maximal (un élément $a$ tel qu'il n'existe pas d'arête $(a,b)$ dans le graphe).

\paragraph{Question. 3\\}
Montrer qu'à la fin de la méthode par rangement des paires, il existe un unique 
élément maximal.

\paragraph{Question. 4\\}
(*) Écrire une fonction \texttt{ajoutable: int -> int -> graphe -> bool} qui étant 
donné deux entiers $i,j$ et un graphe $\mathcal{G}$ dont on suppose qu'il ne 
contient pas l'arête $(i,j)$, dit si l'ajout de l'arête $(i,j)$ à $\mathcal{G}$ 
va créer un cycle. La fonction renverra \texttt{false} dans ce cas.

Pour répondre à cette question on pourra au préalable réfléchir à la question 
1-4c) sur la clôture transitive, si ce n'est pas déjà fait.

\paragraph{Question. 5\\}
On suppose qu'un fonctionnaire a déjà fabriqué le tableau
des préférences, qui contient à l'indice $(A,B)$ le nombre de personnes qui 
pensent que $A<B$. Écrire une fonction \texttt{elit: int vect vect -> int} qui, 
étant donné le tableau des préférences, détermine le vainqueur pour la méthode 
par rangement des paires.

\section{Morphisme de graphes}
On appelle un morphisme de graphe, une application d'un graphe dans un
autre qui préserve l'adjacence.

\paragraph{Question. 1\\} Ecrire une fonction qui prend deux graphes,
une fonction d'un graphe dans l'autre, et détermine si cette fonction est
un morphisme de graphe. 

\paragraph{Question. 2\\}
Qu'est-ce qu'un morphisme de graphes pour le monde des relations d'ordres~?

\paragraph{Facultatif culturel\,:\\}
(*) On dit que deux graphes sont isomorphes s'il 
existe entre eux un morphisme de graphes bijectif dont l'inverse est un 
morphisme de graphes.

Écrire une fonction qui détermine si deux graphes
sont isomorphes. Quelle est la complexité de votre algorithme\,?

\section{Coloriage de graphes}

Un $k$-coloriage d'un graphe $\mathcal{G}$ est un coloriage des sommets du 
graphe $\mathcal{G}$ à
l'aide de $k$ couleurs, tel que deux sommets adjacents ne sont jamais de la
même couleur.

\paragraph{Question. 1\\}
Montrer que tout arbre est 2-coloriable.

\paragraph{Question. 2\\}Le degré d'un sommet est le nombre de voisins
de ce sommet, le degré d'un graphe et le maximum des degrés de ses
sommets.

On va encoder un coloriage par un tableau d'entier, de la taille du
nombre de sommets du graphe. Montrer que tout graphe de degré borné par
$k$, est $k+1$-coloriable. Écrire une fonction qui renvoie un tel coloriage. 

\paragraph{Question. 3\\}
On définit le nombre chromatique du graphe $\mathcal{G}$ comme le plus petit $k$ tel que
$\mathcal{G}$ soit $k$-coloriable. Si c'est le graphe d'une relation d'equivalence, à quoi
correspond ce $k$~?

\section{Une dernière notion~: Circuits hamiltoniens et eulériens}
Ici on va considérer des graphes non-orientés, connexes.
On appelle un circuit hamiltonien (resp. eulérien) un chemin dans un graphe qui passe une
et une seule fois par chaque sommet (resp. arête), et dont le départ et l'arrivée
sont confondus. Contrairement à votre
intuition légitime, on pense
que l'un des deux problèmes est beaucoup plus difficile que l'autre. 

\paragraph{Question. 1\\}
(*) Écrire une fonction \texttt{isHamiltonian} qui
dit si un graphe admet un circuit hamiltonien, et si oui renvoie un tel circuit.
Quelle est la complexité de votre algorithme~? \emph{Culture~: C'est le
bien connu problème du voyageur de commerce.}

\paragraph{Question. 2\\}
On appelle code de Gray une énumération des entiers de $[0,2^n-1]$ par
des mots binaires de $n$ bits, telle que deux entiers successifs ont des
représentants qui ne diffèrent que d'un bit. En utilisant la question
précèdente sur un graphe bien choisi, montrez l'existence d'un tel code
de Gray pour $n=4$. Ceci était une $\epsilon-$introduction à la théorie
des codes correcteurs d'erreurs. C'est \emph{*intéressant*}. Certains
codes de Gray sont simples à calculer~: on pourra prendre 2mn pour se
documenter.

\paragraph{Question. 3\\} Prouver qu'un graphe connexe non orienté possède un circuit
eulérien si et seulement si chaque sommet est de degré pair. 
Écrire une fonction Caml \texttt{isEulerian} qui décide si un graphe possède un
circuit eulérien. 

\end{document}
