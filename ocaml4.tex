\documentclass[10pt,a4paper]{article}
\usepackage[francais]{babel}  %Doc fr
\usepackage{fullpage}
\usepackage{euler}
\usepackage{fontspec}
\usepackage{amsmath}
\usepackage{framed}
\usepackage{amsfonts}
\usepackage{amssymb}
\usepackage{graphicx}
\usepackage{hyperref}

\setmainfont[Numbers=OldStyle]{Linux Libertine O}

\begin{document}
\title{Caml-TD3 : Relations, graphes et votes.}
\author{Thomas Bourgeat\footnote{Vous pouvez bien entendu me contacter par mail thomas.bourgeat@ens.fr, même en cas de début de question : N'HÉSITEZ PAS.}\and Paul Melotti}
\maketitle{}

\subsection{Graphe de relation}

Dans ce sujet on va s'intéresser à visualiser graphiquement des
propriétés de relations. Pour représenter une relation binaire $\mathcal{R}$ sur un ensemble
$X$, on va construire le graphe suivant : 
\begin{itemize}
\item Les sommets sont les éléments de l'ensemble.
\item On construit une arête (orientée) $(s,t)$ lorsque $(s,t)\in
\mathcal{R}$
\end{itemize}
\paragraph{Q.1}
On implémente une relation binaire sur un ensemble fini X, par son graphe
comme définit plus haut. Et on représente en Caml ce graphe par sa
matrice d'adjacence. Construisez le graph correspondant à la relation de
divisibilité entière sur l'ensemble $\{1,2,\dots,9\}$

\paragraph{Q.2}
Décrire quelles propriétés particulières vérifie le graphe lorsque la
relation est :
\begin{itemize}
\item Réflexive
\item Symétrique
\item Transitive
\item Antisymétrique
\end{itemize}  
Ecrire des fonctions qui étant donné un graphe de relation, vérifient si
la relation associée est Réflexive, Symétrique, \dots.

\paragraph{Q.3}


\end{document}
