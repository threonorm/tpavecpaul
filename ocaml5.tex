\documentclass[10pt,a4paper]{article}
\usepackage[francais]{babel}  %Doc fr
\usepackage{fullpage}
\usepackage{euler}
\usepackage{fontspec}
\usepackage{amsmath}
\usepackage{framed}
\usepackage{amsfonts}
\usepackage{amssymb}
\usepackage{graphicx}
\usepackage{hyperref}
\usepackage{enumerate}

%\setmainfont[Numbers=OldStyle]{Linux Libertine O}

\begin{document}
\title{Expressions régulières}
\author{Thomas Bourgeat\footnote{Vous pouvez bien entendu me contacter par mail thomas.bourgeat@ens.fr, même en cas de début de question : N'HÉSITEZ PAS.}}
\maketitle{}
Ce TP est un TP de compilation de Jean-Christophe
Filliâtre  très légèrement modifié pour détailler certains points. 
\\
On va définir les expressions régulières sur un alphabet (donc les lettres
seront des char). Une subtilité par rapport à ce que vous connaissez c'est qu'on
va vouloir indexer les lettres, c'est à dire que l'expression régulière $a.(a+b)$
va devenir $a_1.(a_2+b_1)$. Autrement dit les lettres d'une expression
régulières sont pour ce TD des \texttt{char*int}. Pas de panique sur la
signification de cet indice, c'est simplement une annotation qui servira plus
tard, elle indique par exemple que $a_2$ est le second a en lisant de gauche à
droite.

\paragraph{Question.1\\}
Proposer un type pour les expressions régulières, puis
écrire une fonction \texttt{val null : regexp -> bool} qui détermine si
$\epsilon$ appartient au langage reconnu par une expression régulière.

\paragraph{Question.2\\}

\subparagraph{a.}On va avoir besoin de manipuler des ensembles de lettres indexées, on va donc
les représenter par des listes : \texttt{val : (char*int) list}. Faire rapidement les fonctions classiques sur
les ensembles :
\begin{itemize}
\item Presence dans l'ensemble
\item Ajout d'element
\item Suppression d'element (si présent)
\item Union d'ensembles
\item Intersection d'ensembles
\end{itemize}
L'utilisation de listes pour représenter de tels ensembles est essentiellement
pratique parce que facile à implémenter, mais les complexités sont pas
optimales.
Quelle structure de donnée vous pourriez utiliser ? Justifiez rapidement en
comparant avec
quelques arguments de complexités  (Je ne vous demande pas de l'implémenter).
\\

\subparagraph{b.}
Écrire une fonction \texttt{val first : regexp - >  (char*int) list}
qui calcule l'ensemble des premières lettres des mots reconnus par une
expression régulière. (Il faut se servir de \texttt{val : null}).
De même écrire une fonction \texttt{val last : regexp -> (char*int) list} 
\\ Conseil universel, spécifié pour cette question : Réfléchissez sur papier,
que ce soit clair, avant de commençer à coder.

\paragraph{Question.3\\}
En se servant de ce qui précède, écrire \texttt{val follow : (char*int) ->
regexp-> (char*int) list} qui calcule l'ensemble des lettres qui peuvent suivre
une lettre donnée dans l'ensemble des mots reconnus.
On remarquera notamment que \texttt{d} est dans \texttt{follow c r} si et
seulement si :
\begin{itemize}
\item soit il existe une sous-expression $r=r1.r2$ avec \texttt{d} élément de
\texttt{first r2} et c élément de \texttt{last r1}
\item soit il existe une sous-expression $r=r1^{*}$ avec d élément de
\texttt{first r1}  et c élément de \texttt{last r1} 
\end{itemize}

\paragraph{Question.4\\}
Pour tester si un mot est dans un langage définit par une expression régulière
on va utiliser un outils que vous allez voir bientot (avez-vu?) : \emph{les
automates}. Bon tout d'abord quelques précisions :
\begin{itemize}
\item Le but du TP est de vous donner une première intuition de cette notion,
même si c'est avant le cours : pédagogie par l'immersion. Ainsi durant votre cours,
il sera important d'essayer de visualiser chacune des notions qui sera
présentée. Le but étant d'être à l'aise avec la notion d'automates et son lien
étroit avec la notion de langages rationnel.
\item Ce qui va être présenté ici n'est pas nécéssairement exactement dans la
même forme que ce qui sera présenté en cours.
\item Essentiellement le sujet établi un des sens du théorème de Kleene que vous
allez certainement revoir en cours d'une manière différentes : \emph{Les
langages reconnaissables par automates finis sont exactement les langages
rationnels}.
\item Pour ceux qui veulent aller plus loin, je conseil le livre \emph{Langage formels} d'Olivier
Carton. 
\end{itemize} 

\hrulefill

\emph{Moment de présentation des automates au tableau, si vous ne connaissez pas
déjà.}

\hrulefill

Pour construire l'automate déterministe correspondant à une expression
régulière r (un état est un ensemble de lettres indéxées) on fait :
\begin{itemize}
\item On ajoute un nouveau caractère $\sharp$ à la fin de r. (représenté
('$\sharp$',-1)).
\item L'état de départ est \texttt{first r}.
\item On a une transition d'un état q à q' à la lecture du caractère c (ici le
caractère est non indexé), si q' est l'union de tous les \texttt{follow ci r}
pour tous les éléments ci de q tels que \texttt{fst ci=c}.
\item les états d'acceptation sont ceux qui contiennent le caractère $\sharp$.
\end{itemize}
Ne paniquez pas si vous ne comprenez pas tout à la première lecture, laissez
vous guider par les questions.

\subparagraph{a.} Écrire une fonction \texttt{val nextstate : regexp ->
(char*int) list-> char-> (char*int) list} qui cacule l'état résultant d'une
transition.
Pour représenter un automate on va avoir besoin de listes associatives
(\texttt{type ('a,'b) assoc = (a*b) list;;}, encore
une partie facile de TP ou il faut aller vite : on rappelle qu'une liste
associative est une liste de la forme \texttt{(clef,elements) list}, un peu
comme un dictionnaire. C'est une version simple des tables de hachage (table de
hachage dans un tableau à une seule case). imaginez le comme un dictionnaire, ou
vous cherchez selon le nom du mot (la clef), et ça vous renvoit la définition
(l'élément). Ecrire les fonctions
suivantes : 
\begin{itemize}
\item Ajout d'un élément à une liste associative
\item Suppression d'un élément 
\item Recherche par clef.
\end{itemize}
 
 Pour représenter un automate on va utiliser le type :
\\\texttt{
type autom = \{ \\
start : (char,int) list;\\
transit : ((char,int) list, ((char, (char,int) list ) assoc)  assoc;\\
\}
}\\
Encore une fois pas de panique, le type à l'air compliqué, mais mettez des vrais
noms concrets sur ça, et vous allez voir que c'est naturel!
\\
\subparagraph{b.}
Écrire une fonction \texttt{val makeDfa: regexp -> autom} qui construit
l'automate correspondant à une expression régulière. L'idée est de construire
les états par nécessité, en partant de l'état initial. On pourra par exemple
adopter l'approche suivante :
\\
\texttt{
let makeDfa r = \\
  let r = Concat (r, ('$\sharp$',-1) ) in\\
  (* transitions en cours de construction *)\\
  let trans = ref \[ \] in\\
  let rec transitions q =\\
    (* la fonction transitions construit toutes les transitions de l'état q,\\
       si c'est la première fois que q est visité *)\\
    ...\\
  in\\
  let q0 = first r in\\
  transitions q0;\\
  \{ start = q0; trans = !trans \}\\
}

\end{document}
