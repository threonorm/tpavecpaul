\documentclass[10pt,a4paper]{article}
\usepackage[francais]{babel}  %Doc fr
\usepackage{fullpage}
\usepackage{euler}
\usepackage{fontspec}
\usepackage{amsthm}
\usepackage{amssymb}
\usepackage{mathrsfs}
\usepackage{amsmath}
\usepackage{amsfonts}
\usepackage{framed}
\usepackage{graphicx}
\usepackage{hyperref}
\usepackage{enumerate}
\usepackage{qtree}

%\setmainfont[Numbers=OldStyle]{Linux Libertine O}
\theoremstyle{definition}
\newtheorem*{Ex}{Exemple}

\begin{document}
\title{TP7 : Syntaxe et sémantique}
\author{Paul Melotti (\texttt{paul.melotti@ens.fr})}
\maketitle{}

Le but de ce TP est de comprendre l'intérêt des arbres syntaxiques et leur
interprétation. Le cadre est le suivant : on dispose d'un ensemble de variables
$V$, et d'un ensemble de symboles fonctionnels $\Sigma$. Un symbole $f \in \Sigma$
est caractérisé par son \textit{arité} $\alpha (f) \in \mathbb{N}$, que vous 
pouvez voir comme le ``nombre d'arguments'' de $f$. Un symbole $f\in \Sigma$
d'arité 0 est appelé symbole constant.

Une \textit{expression} sera un arbre étiqueté par $\Sigma \cup V$ tel que :
\begin{itemize}
\item les feuilles sont étiquetées par des variables ou des syboles constants
\item les noeuds ayant $n$ fils sont étiquetés par des symboles d'arité $n$.
\end{itemize}

\begin{Ex}
Pour représenter des expressions logiques en les variables $A$ et $B$,
on prend :
\begin{itemize}
\item Symboles d'arité 0 : $\{ \mathrm{Vrai}, \mathrm{Faux}\}$
\item Symboles d'arité 1 : $\{ \mathrm{not} \} $
\item Symboles d'arité 2 : $\{ \wedge , \vee , \Rightarrow , \Leftarrow , \Leftrightarrow \} $
\item Variables : $\{ A, B \}$.
\end{itemize}
Alors l'arbre suivant est une expression logique ; quelle formule logique 
voudriez-vous qu'elle représente ?

\Tree[.$\Leftrightarrow$ [.$\Rightarrow$ [.not [.A ] ] [.not [.B ] ] ]
         [.$\Rightarrow$ [.B ] [.A ] ] ]
\end{Ex}

\section{Calcul formel}
On va utiliser ces arbres pour faire du calcul formel, comme le fait
peut-être votre calculatrice. Pour simplifier, on se limitera aux symboles 
suivants :
\begin{itemize}
\item Symboles d'arité 0 : les flottants
\item Symboles d'arité 1 : $\{ \exp \} $
\item Symboles d'arité 2 : $\{ +, * \} $
\item Variables : les caractères.
\end{itemize}
Pour manipuler ces arbres, on définit le type suivant :
\begin{verbatim}type expr = Var of char
            | Const of float
            | Exp of expr
            | Somme of expr*expr | Produit of expr*expr ;;\end{verbatim}
\paragraph{Question.1} Écrire une fonction \texttt{calcule : expr -> float} qui
prend en argument une expression et renvoie sa valeur. On déclenchera une 
exception lorsqu'une variable apparaît dans l'expression.

\paragraph{Question.2} Écrire une fonction \texttt{evalue : expr -> (char -> float) -> float}
qui prend non seulement une expression, mais aussi une fonction qui dit ce que 
valent les variables, et renvoie la valeur associée à l'expression.

\paragraph{Question.3} Écrire une fonction \texttt{derive : char -> expr -> expr} qui 
dérive une expression par rapport à la variable donnée en argument.

\paragraph{Question.4} Que renvoie \texttt{derive `x` (Exp (Somme (Var `x`,Const 1.)))} ?
Écrire une fonction \texttt{simplifie : expr -> expr} qui effectue quelques
simplifications sur une expression.
\\

Dans cette partie, on est passé de la \textit{syntaxe} (façon d'organiser et de
représenter les objets) à la \textit{sémantique} (liée au sens, interprétation
de cette écriture). Les objets syntaxiques comme le symbole \texttt{Exp} 
pourraient représenter n'importe quoi ; ici, on l'a \textit{interprété} comme
la fonction exponentielle des réels. Par exemple, les expressions 
\texttt{Exp(Const(0.))} et \texttt{Const(1.)} sont deux objets syntaxiques
différents, mais leur interprétation est ici la même.

Dans la partie suivante, on va faire quelque chose de purement syntaxique.

\section{Syntaxe concrète}
Vous avez dû vous rendre compte que les arbres syntaxiques sont bien utiles
pour le programmeur, mais pas très pratiques pour l'utilisateur qui n'a pas
envie d'écrire tout les temps des choses du style \texttt{Exp (Somme (Var `x`,Const 1.))}.
On va ici voir comment passer d'écritures plus naturelles à la structure 
d'arbres.

On se placera dans le cas général suivant : on travaille avec le type
 
\texttt{type expr = C of string | U of string*expr | B of string*expr*expr}

où les \texttt{C} sont les constantes, les \texttt{U} (unaire) sont les 
symboles d'arité 1, et les \texttt{B} (binaire) les symboles d'arité 2. Le
\texttt{string} sert simplement à donner un nom à ces différents symboles, par exemple,
on écrira \texttt{U("exp", B("+", C "x", C "1") )}. Remarquez qu'on n'a pas 
différencié les constantes des variables, ce qui ne posera pas de problème dans
ce qu'on va faire.

\subsection{Forme postfixe}
On appelle écriture postfixe d'une expression $t$ le mot de $\Sigma^*$  défini 
inductivement par 
\[Pos(t)=\left\lbrace
\begin{array}{lcl}
t & & \mathrm{si} \ t \ \mathrm{est constant}\\
Pos(t_1)\dots Pos(t_n) f & & \mathrm{si} \ t = f(t_1,\dots t_n)
\end{array}\right.\]

\paragraph{Question.1} Écrire sous forme postfixe l'exemple d'arbre syntaxique 
précédent.

\paragraph{Question.2} Écrire une fonction \texttt{ecritpost : expr -> unit} qui
imprime l'écriture postfixe d'une expression quelconque.
\\

On va maintenant s'intéresser au problème inverse : étant donné une écriture
postfixe, comment retrouver l'arbre syntaxique associé ?
\\

Une méthode possible est la suivante : on utilise une pile $p$ initialement vide.
On lit l'un après l'autre les symboles du mot :
\begin{itemize}
\item à chaque symbole constant, on l'empile
\item à chaque symbole $f$ d'arité $n\geq 1$, on dépile successivement $t_n,\dots, t_1$
(attention, l'ordre compte !) et on empile $f(t_1, \dots, t_n)$.
\end{itemize}

Il est possible de montrer\footnote{on ne demande pas de le faire. Mais c'est 
un bon exercice, plutôt difficile...} que si le mot 
correspondait à $Pos(t)$, alors l'algorithme ne déclenche pas d'erreur et qu'à la
fin il ne reste dans la pile qu'un seul élément, l'expression $t$. Réciproquement, si 
l'algorithme appliqué à un mot $u$ ne déclenche pas d'erreur et finit avec un 
seul élément élément dans la pile, alors $u$ était bien une écriture postfixe.

\paragraph{Question.3} On représente le mot par une \texttt{int*string list} :
les éléments de cette liste sont les symboles dans $\Sigma$, représentés sous
forme de couples où l'entier indique leur arité (0,1 ou 2) et la chaîne de caractère leur
nom. Par exemple, \texttt{[(0,"1");(0,"x");(2,"+")]} représente le mot \texttt{1 x +}.

Écrire une fonction \texttt{arbre\_de\_liste : int*string list -> expr} qui 
calcule l'expression associée à cette écriture, grâce à l'algorithme précédent.

\paragraph{Question.4 - Facultative} Cette écriture sous forme de liste n'est pas non plus
très satisfaisante ; on aimerait bien pouvoir écrire carrément des chaînes de 
caractères. Le problème qui consiste à transformer une chaîne de caractère
en, mettons, la liste précédente, est un problème de \textit{parsing}. Dans cette
petite introduction, nous utiliserons la convention suivante : les noms des
constantes seront introduits par la lettre \texttt{C}, les fonctions unaires
par la lettre \texttt{U} et les fonctions binaires par la lettre \texttt{B}. Ces
trois lettres seront interdites dans les noms des fonctions et constantes.
Par exemple, la chaîne \texttt{"C1CxB+"} correspond à l'exemple de la question
précédente.

Écrire une fonction \texttt{parse : string -> int*string list} qui transforme
cette chaîne de caractère en la liste associée (on ne gardera pas les \texttt{C},
\texttt{U} et \texttt{B} dans les noms), puis une fonction 
\texttt{arbre\_de\_chaine : string -> expr} qui transforme une telle écriture en
son arbre syntaxique.
\begin{Ex}
\begin{verbatim} #arbre_de_chaine "CxC1B+Uexp";;
- : expr = U ("exp", B ("+", C "x", C "1"))\end{verbatim}
\end{Ex}

\subsection{Forme préfixe}
L'écriture préfixe d'une expression $t$ est définie inductivement par :
\[Pre(t)=\left\lbrace
\begin{array}{lcl}
t & & \mathrm{si} \ t \ \mathrm{est constant}\\
f \ Pre(t_1)\dots Pre(t_n) & & \mathrm{si} \ t = f(t_1,\dots t_n)
\end{array}\right.\]

\paragraph{Question.6} La forme préfixe est-elle l'image miroir de la forme 
postfixe ?

\paragraph{Question.7} Refaire la question 3 dans le cas où la liste représente
un mot sous forme préfixe.

\textit{Indication :} on pourra écrire une fonction récursive qui prend en 
argument une \texttt{int*char list} et renvoie l'arbre syntaxique associé au 
début de la liste, ainsi que le reste de la liste.

\section{Expressions rationnelles sur un alphabet $X$}
Retour à l'interprétation sémantique, et à des choses que vous connaissez : on 
veut manipuler des expressions rationnelles. On va donc utiliser les symboles 
suivants :
\begin{itemize}
\item Symboles d'arité 0 : $X \cup \{ \epsilon \}$
\item Symboles d'arité 1 : $\{ *, + \} $
\item Symboles d'arité 2 : $\{ .\} $
\item ...Pas de variables !
\end{itemize}

\paragraph{Question.1} Proposer un type \texttt{regexp} pour les expressions 
rationnelles. On pourra prendre $X$ l'ensembles des \texttt{char}.

\paragraph{Question.2} Écrire une fonction \texttt{avide : regexp -> bool} qui 
détermine si $\epsilon$ appartient au langage.

\paragraph{Question.3} Écrire une fonction \texttt{arden : regexp -> regexp -> regexp}
qui étant donné deux expressions $A$ et $B$, avec $\epsilon \notin A$, résout
l'équation $X=A.X + B$. Si $\epsilon \in A$ on renverra une erreur.

\paragraph{Question.4} On considère le système d'inconnues les langages $X_1,...X_n$
suivant :
\[\left\lbrace
\begin{array}{lcl}
X_1 &=& A_{1,1}.X_1 + A_{1,2}.X_2 + \dots + A_{1,n}.X_n + B_1\\
\dots \\
X_n &=& A_{n,1}.X_1 + A_{n,2}.X_2 + \dots + A_{n,n}.X_n + B_n
\end{array}\right.\]
où on suppose que pour tous $i,j$, $\epsilon \notin A_{i,j}$.

Écrire une fonction \texttt{syst : regexp vect vect -> regexp vect -> regexp vect}
qui prend en argument la matrice des $A_{i,j}$ et le vecteur des $B_i$, et résout
le système.

\paragraph{Question.5} Comment la fonction précédente permet-elle de calculer 
le langage reconnu par un automate ?

\end{document}
