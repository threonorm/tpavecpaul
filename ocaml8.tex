\documentclass[10pt,a4paper]{article}
\usepackage[francais]{babel}  %Doc fr
\usepackage{fullpage}
\usepackage{euler}
\usepackage{fontspec}
\usepackage{amsmath}
\usepackage{framed}
\usepackage{amsfonts}
\usepackage{amssymb}
\usepackage{graphicx}
\usepackage{hyperref}
\usepackage{enumerate}

%\setmainfont[Numbers=OldStyle]{Linux Libertine O}

\begin{document}
\title{TP8: Programmation dynamique}
\author{Thomas Bourgeat \and Paul Melotti \texttt{(paul.melotti@ens.fr)}}
\date{6 et 13 février 2014}
\maketitle{}

\section{Multiplications matricielles optimales}
Soit $A_1, \dots, A_n$ des matrices rectangulaires, telles que
\[A_1\in \mathcal{M}_{p_0,p_1}, A_2\in \mathcal{M}_{p_1,p_2}, \dots, A_n\in \mathcal{M}_{p_{n-1},p_n}.\]
On cherche à calculer le produit $A_1 \times \dots \times A_n$.
On dispose du tableau $p = [|p_0 ; \dots p_n |]$ des tailles successives.

\paragraph{Question.1}

Combien de multiplications faut-il effectuer pour calculer le produit $A B$
où $A\in \mathcal{M}_{p,q}$ et $B\in \mathcal{M}_{q,r}$ ?

Par exemple, si $p=[|10;100;5;50|]$, combien faut-il d'opérations pour calculer
le produit si on le parenthèse en $(A_1 \times A_2) \times A_3$ et en 
$A_1 \times (A_2 \times A_3)$ ?
\\

On constate que l'ordre dans lequel on effectue les multiplications est 
important. L'objectif de cette partie est de trouver le parenthésage optimal pour 
calculer le produit.

Si $i \leq j$, on note $m_{ij}$ le nombre minimal de multiplications scalaires 
pour calculer le produit $A_i \dots A_j$, et $M$ la matrice des
$m_{ij}$ (qui est triangulaire supérieure de diagonale nulle).

Le parenthésage optimal de ce calcul sépare le produit en 
$A_i \dots A_j = (A_i \dots A_k) (A_{k+1} \dots A_j)$ pour un certain indice de
coupure $k\in [i;j-1]$. On note $s_{ij}$ ce $k$, et $S$ la matrice des $s_{ij}$.

\paragraph{Question.2} Montrer que 
\[m_{ij} = \left\{
	\begin{array}{ll}
		0  & \mbox{si } i=j \\
		\min_{i\leq k < j} (m_{ik} + m_{k+1,j} + p_{i-1} p_k p_j) & \mbox{si i < j}.
	\end{array}
\right.\]
Cette formule permet de remplir la matrice $M$ intelligemment si on calcule les
coefficients dans un certain ordre, lequel ?

\paragraph{Question.3} Écrire une fonction \texttt{couts : int vect -> int vect vect}
qui étant donné le vecteur \texttt{p} renvoie la matrice \texttt{M}. Quelle est
sa complexité ?

\paragraph{Question.4} Modifier la fonction précédente en une fonction 
\texttt{coupes : int vect -> int vect vect} qui étant donné \texttt{p} renvoie
la matrice \texttt{S}.

\paragraph{Question.5} Écrire une fonction \texttt{parenthesages : int vect -> unit}
qui étant donné \texttt{p} imprime le parenthésage optimal. Par exemple :
\begin{verbatim}#parenthesages [|3;2;1;4;2;2|] ;;
((1)(2))(((3)(4))(5))- : unit = () \end{verbatim}

\paragraph{Question.6 (*)} Combien y a-t-il de façons de parenthéser le produit
$A_1 \dots A_n$ ? Quelle serait la complexité d'un algorithme qui énumèrerait
tous les parenthésages et calculerait le coût de chacun ?

\section{Un peu de réflexion autour de la programmation dynamique}
L'idée de la programmation dynamique est de pouvoir résoudre un problème
d'\emph{optimisation} en
commençant par résoudre des sous-problèmes et en recombinant les solutions de
ces sous-problèmes. 

Pour que cela soit possible, il faut que le problème ait une structure
particulière. Par exemple s'il vérifie la propriété de sous-structure optimale :
lorsqu'on a une solution optimale à notre problème, cette solution contient des
solutions optimales à des sous-problèmes.
Par exemple, pour la section précèdente, trouver un bon parenthèsage pour
$A_1.\dots.A_n$ nous donne un bon parenthèsage pour un couple de 
$A_1.\dots.A_i$ et $A_{i+1}.\dots.A_n$ .

Autrement dit, la solution au gros problème donne des solutions à des petits
problèmes dont il est composé.

Souvent, les stratégies de programmation dynamique consistent à grossir un peu 
le problème : dans la section précédente, on a trouvé des parenthésages optimaux
pour plein de couples $(i,j)$ inutiles en pratique.

\paragraph{Question.0 (orale)} Parfois les étudiants confondent récursivité et
programmation dynamique. Pensez-y, pourquoi est-ce que ce n'est pas la
même chose, pourquoi est-ce qu'il y pourrait y avoir confusion?

\paragraph{Question.1} Avez-vous des idées de problèmes qui pourraient se 
traiter par programmation dynamique? De problèmes ne s'y prêtant pas ?

\subsection{Dynamique contre glouton}
On s'intéresse au problème du rendu de monnaie~: on dispose de pièces
(on en a autant qu'on veut !) de valeurs suivantes : $\{p_1,\dots,p_n\}$, où les
$p_i$ sont triées par ordre croissant. On cherche le nombre minimum de 
pièces qu'il faut pour payer une somme $s$.

Une première idée est d'utiliser un algorithme \textit{glouton} : on essaie de
payer $s$ en mettant la plus grande pièce possible $p_i$, puis on est ramené à
payer $s-p_i$ et on recommence. De manière générale, un algorithme glouton est
un algorithme qui tente à chaque étape d'utiliser un optimum local (\textit{i.e.}
de manger le plus gros morceau possible).

\paragraph{Question.2} Écrire une fonction \texttt{glouton : int list -> int -> int}
qui tente de résoudre le problème du rendu de monnaie par l'algorithme glouton
(on reçoit en entrée la liste des pièces disponibles et la somme à payer, et on 
renvoie le nombre de pièces utilisées pour payer).
Quelle est sa complexité ?
\paragraph{Question.3} La fonction précédente est-elle optimale ? Par exemple,
si je dois payer $6$ et que les pièces disponibles sont $1,3$ et $4$ ?
\paragraph{Question.4} Résoudre le problème de façon optimale par programmation
dynamique. Quelle est la complexité de votre fonction ?\\
\textit{Indication :} on pourra calculer le nombre de pièces optimal pour toutes
les sommes $i\leq s$.

\subsection{Facultatif - La revanche du glouton}
On s'intéresse au problème de la gestion d'une salle : $n$ personnes
veulent organiser des événements dans notre salle avec des horaires de début 
$d_i$ et de fin $f_i$ pour chaque événement. Mais certains horaires sont
incompatibles. On souhaite maximiser le nombre d'événements dans notre salle (et
non pas le temps d'occupation).

\paragraph{Question.5} On propose plusieurs algorithmes gloutons pour répondre 
au problème. Dites si ces idées donnent une solution optimale; si oui, donnez 
une preuve, et si non, donnez un contre-exemple\footnote{Conseil : cherchez
d'abord des contre-exemples}.
\begin{enumerate}[a)]
\item Mettre à chaque fois l'événement le plus court parmi les événements 
compatibles.
\item Mettre à chaque fois l'événement qui commence le plus tôt parmi les 
événements compatibles.
\item Mettre à chaque fois l'événement qui finit le plus tôt parmi les 
événements compatibles.
\end{enumerate}

\paragraph{Question.6} Écrire une fonction \texttt{salle : int vect -> int vect -> int}
qui étant donné les tableaux des $d_i$ et des $f_i$ (les $i$ étant dans un ordre
quelconque) donne le nombre maximal d'événements qu'on peut organiser dans notre
salle, en utilisant l'algorithme glouton approprié. Quelle est sa complexité ?

\paragraph{Question.7} Voyez-vous comment résoudre ce problème par programmation
dynamique ? Écrivez sur papier une idée d'algorithme. Quelle serait sa 
complexité ?

\section{Quelques exercices faisables en dynamique}

\subsection{Fibonacci (encore...)}
Dans le cas du calcul du $n$-ième terme de la suite de Fibonacci, en quoi 
consisterait une approche dynamique ? Coder
cette fonction.

\subsection{Problème du sac à dos}
On dispose de $n$ objets de poids $p_1,...,p_n$ et de valeur (pécuniaire) 
$v_1,...,v_n$. On souhaite remplir un sac à dos de manière à transporter la plus
grande valeur possible, mais il ne peut contenir qu'un poids inférieur à $P$.
Trouver un algorithme qui donne la valeur maximale qu'on peut transporter.

\textit{Indication :} On pourra s'intéresser à $V_{i,p}$ la valeur maximale
qu'on peut transporter si on n'a que les $i$ premiers objets et un sac à dos de
contenance $p \leq P$.

\subsection{Problème de gain}
Énoncé de Louis Jachiet : \\
\textit{Après avoir voyagé dans le futur puis être revenu à votre
époque, vous disposez d’un tableau qui vous dit ce que
vaut, chaque mois, le cours de l’or. Comme vous vou-
lez à tout prix vous enrichir, mais en faisant le moins
d’efforts possible, vous décidez de regarder quel est le
profit maximal que vous puissiez faire. Le cours de l’or
est présenté sous la forme d’une liste d’entiers et vous
devez donner le gain maximal réalisable. Vous ne pouvez
faire qu’un achat/vente car sinon vous modifieriez trop
le cours du temps et donc cela changerait les cours (vous
pourriez y perdre) mais cela risquerait aussi de modifier
le continuum espace/temps lui même !}

Vous cherchez donc le gain maximum, qui s'exprime par
\[G = \max_{0\leq i \leq j \leq n-1} (v_j - v_i). \]
\paragraph{Question.1} $G$ est-il toujours égal à la différence entre le maximum et le minimum du tableau ?
\paragraph{Question.2} Écrire une fonction efficace qui détermine $G$.

\textit{Indication :} $g_i = \max_{0\leq k \leq i} (v_i - v_k)$
\paragraph{Question.3} Donner un algorithme qui détermine de manière efficace la plus grande somme formée par des éléments consécutifs d'un tableau de nombres.

\subsection{Sous-chaîne commune}
Écrire un algorithme qui trouve la plus longue sous-séquence commune 
à deux chaînes de caractères.

\subsection{(**) Points fixes d'une itérée de fonction affine par morceaux}
Soit $m$ un entier, et $f$ une fonction de $[0;m]$ dans $[0;m]$ telle que :
\begin{itemize}
\item Pour tout entier $i\leq m$, $f(i)$ est un entier noté $f_i$ ;
\item Pour tout entier $i<m$, $f$ est affine sur le segment $[i;i+1]$.
\end{itemize}
Donner une algorithme qui, étant donné le tableau des $f_i$ et un entier $k$, détermine le nombre de points fixes de $f^k$ (l'itérée $k$-ième de $f$). Votre algorithme devra être au plus linéaire en $k$.


\end{document}
