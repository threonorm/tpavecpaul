\documentclass[10pt,a4paper]{article}
\usepackage[francais]{babel}  %Doc fr
\usepackage{fullpage}
\usepackage{euler}
\usepackage{fontspec}
\usepackage{amsmath}
\usepackage{framed}
\usepackage{amsfonts}
\usepackage{amssymb}
\usepackage{graphicx}
\usepackage{hyperref}
\usepackage{enumerate}

%\setmainfont[Numbers=OldStyle]{Linux Libertine O}

\begin{document}
\title{TP9: Labyrinthes}
\author{Paul Melotti (\texttt{paul.melotti@ens.fr}\footnote{N'HÉSITEZ à 
m'écrire pour toute question sur l'informatique, les TIPE, les concours... Je 
reste à votre disposition.})}
\date{6 et 13 mars 2014}
\maketitle{}

L'objectif de ce TP est de découvrir comment les notions de parcours en 
profondeur et en largeur que vous avez vues peuvent être utilisées dans le cas
des labyrinthes.

Un labyrinthe sera représenté de la façon suivante : c'est une matrice d'entiers,
qui valent 0 sur les zones dégagées et 1 sur les murs. \textbf{On suppose que tous
les coefficients du bord de la matrice sont des 1.} Une seule case de la matrice
contient l'entier 2, c'est la sortie. On a le droit de se déplacer dans les quatre
directions, mais pas en diagonale. Exemple :
\begin{verbatim}let laby = [|
  [|1;1;1;1;1;1;1;1;1|];
  [|1;0;1;0;0;1;2;1;1|];
  [|1;0;1;0;1;0;0;0;1|];
  [|1;0;0;0;0;0;1;0;1|];
  [|1;1;1;1;1;1;1;1;1|]
|];;\end{verbatim}
 
\paragraph{Question 1.} Écrire une fonction \texttt{affiche : int vect vect -> unit}
qui affiche un labyrinthe : on représentera les murs par des \texttt{\#}, les 
zones dégagées par des espaces vides et la sortie par un \texttt{S}. Exemple :
\begin{verbatim}affiche laby;;
#########
# #  #S##
# # #   #
#     # #
#########\end{verbatim}

\section{Parcours en profondeur}
On lance un petit lutin dans le labyrinthe à la case $i_0,j_0$ et on le charge
de trouver la sortie. On lui donne un petit pot de peinture, comme ça il peut 
marquer les cases déjà rencontrées. Naturellement, notre lutin va effectuer un
parcours en profondeur :
\begin{verbatim}PP(case) =
   SI c'est la sortie, c'est gagné.
   SI la case n'est pas marquée,
      marquer la case,
      pour toutes les cases voisines,
         PP(voisine).
\end{verbatim}
\paragraph{Question 2.} Pourquoi faut-il marquer les cases déjà rencontrées ?

En Caml, on marquera les cases rencontrées en mettant un 3 dans la matrice.
Comme on ne veut pas modifier notre labyrinthe initial, on a besoin de la 
fonction suivante :
\paragraph{Question 3.} Écrire une fonction \texttt{copie\_matrice} qui renvoie
une copie de la matrice donnée en argument.

\paragraph{Question 4.} Écrire une fonction 
\texttt{estsoluble : int vect vect -> int -> int -> bool} qui prend en entrée un
labyrinthe et la case d'où on part et dit si on peut atteindre la sortie, en 
effectuant un parcours en profondeur.
\\

On a maintenant un petit lutin qui sait nous dire si on peut sortir du 
labyrinthe. Le problème, c'est qu'il ne nous dit pas quel chemin il faut 
prendre. Pour aussi connaître ce chemin, on va écrire sur chaque case visitée
un chemin du départ à cette case, sous la forme d'une \texttt{(int * int) list}.

\paragraph{Question 5.} Reprendre la fonction de la question 4 pour qu'elle 
renvoie un \texttt{bool * (int*int) list} : le booléen dit si on peut sortir,
et la liste nous donne un chemin pour sortir (et ce que vous voulez si on ne 
peut pas sortir).

\section{Parcours en largeur}
On voudrait maintenant connaître le plus court chemin de l'entrée à la sortie.
Pour cela, au lieu de s'enfoncer le plus possible dans les chemins comme avec
un parcours en profondeur, on va plutôt regarder d'abord les cases qui sont
à distance 1, puis celles qui sont à distance 2, etc. C'est l'idée du parcours
en largeur. Pour cela, on va utiliser une file d'attente (structure FIFO) pour
stocker les cases à explorer (et on a donc besoin d'un lutin plus intelligent).

\paragraph{Question 6.} Écrire rapidement des fonctions de manipulation de files
d'attente : enfilage, défilage, et test de file vide.
\\

On utilise la fonction d'exploration suivante : la file d'attente \texttt{f} est
initialisée à \texttt{[(i0,j0)]} et la case $(i_0, j_0)$ est initialement
marquée, et on lance le parcours en largeur :
\begin{verbatim}TANT QUE f non vide,
   Défiler x de f,
   Pour tout voisin y de x,
      SI y non marqué
         enfiler y,
         .........,
         marquer y.
FIN TANT QUE     
\end{verbatim}
La partie laissée en petits points correspond à une certaine action que vous
devrez trouver vous-même.

\paragraph{Question 7.} Se convaincre que cet algorithme (le parcours en largeur)
explore bien les cases par distance à l'origine croissante. 

\textit{Conseil :} essayer sur un exemple.

\paragraph{Question 8.} Écrire une fonction \texttt{voisins : int -> int -> int vect vect -> (int * int) list} qui étant donné $i$, $j$ et un labyrinthe renvoie la liste 
des voisins de la case $(i,j)$ qui ne sont pas marqués et ne sont pas des murs.

\paragraph{Question 9.} Écrire une fonction qui fait comme à la question 5, mais
cette fois-ci renvoie un chemin de longueur minimale lorsque le labyrinthe est
résoluble.

\end{document}
