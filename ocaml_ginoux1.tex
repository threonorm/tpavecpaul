
\documentclass[10pt,a4paper]{article}
\usepackage[francais]{babel}  %Doc fr
\usepackage{fullpage}
\usepackage{euler}
\usepackage{fontspec}
\usepackage{amsmath}
\usepackage{framed}
\usepackage{amsfonts}
\usepackage{amssymb}
\usepackage{graphicx}
\usepackage{hyperref}


\begin{document}
\title{Quelques exos du TD 1 Ginoux}
\maketitle{}

\section{Que diable se passe-t-il dans cette mémoire ?}
Pour une chaîne de caractères $u$ on note $\tilde{u}$ sa chaîne miroir (par exemple si $u = "abcd"$, $\tilde{u} = "dcba"$).

\paragraph{1.} Écrire la fonction qui à $u$ associe $\tilde{u}$.

\paragraph{2.} Écrire la fonction qui teste si un mot est un palindrome, i.e. si $u = \tilde{u}$.

\paragraph{3.} Que penser de la solution suivante~?
\begin{verbatim}let miroir u = let n=string_length u in
    for i=0 to n/2-1 do
    let m=u.[i] in u.[i]<-u.[n-1-i] ; u.[n-1-i]<-m
    done ;
    u ;;
    
let palindrome u = let v = miroir u in u=v ;;    
\end{verbatim}

\section{Typage}
On considère la fonction
\begin{verbatim}let egal x y = match y with
    | x -> true
    | _ -> false ;;
\end{verbatim}
Quel est son type~? Que renvoie \texttt{egal 1 2}~?
\end{document}
